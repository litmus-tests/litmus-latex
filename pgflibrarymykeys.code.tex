%%%%%%%%%%%%%%%%%%%%%%%%%%%%%%%%%%%%%%%%%%%%%%%%%%%%%%%%%%%%%%%%%%%%%%
%                                                                    %
%         rmem executable model                                      %
%                                                                    %
%   Shaked Flur, University of Cambridge                             %
%   Jon French, University of Cambridge                              %
%   Kathryn Gray, University of Cambridge (when this work was done)  %
%   Robert Norton-Wright, University of Cambridge                    %
%   Christopher Pulte, University of Cambridge                       %
%   Susmit Sarkar, University of St Andrews                          %
%   Peter Sewell, University of Cambridge                            %
%                                                                    %
%  This file is copyright 2010-2017 by the authors above.            %
%                                                                    %
%  All rights reserved.                                              %
%%%%%%%%%%%%%%%%%%%%%%%%%%%%%%%%%%%%%%%%%%%%%%%%%%%%%%%%%%%%%%%%%%%%%%

% This library should be loaded using '\usepgflibrary{mykeys}'

\pgfkeys{
  % #1 - package name
  % #2 - warning/error message
  % #3 - help message (as in "Type H <return> for immediate help")
  /utils/package warning/.code 2 args={\PackageWarning{#1}{#2}},
  /utils/package error/.code n args={3}{\PackageError{#1}{#2}{#3}},
}

\pgfkeys{
  % <key>/.get key parts=<macro>
  % Defines <macro> to be fully qualified <key>, '<macro>name' to be
  % the key name of <key> and '<macro>path' to be the path of <key>
  % (i.e. <macro>path/<macro>name = <macro>).
  % For example, after '\tikzset{foo/.get key parts=\mykey}' \mykey expands
  % to '/tikz/foo', \mykeyname expands to 'foo' and \mykeypath expands
  % to '/tikz'.
  /handlers/.get key parts/.code={%
    \edef#1{\pgfkeyscurrentpath}%
    \edef\pgfkeyscurrentkey{\pgfkeyscurrentpath}%
    \pgfkeyssplitpath{}%
    \count@\escapechar \escapechar\m@ne%
      \expandafter\edef\csname\string#1name\endcsname{\pgfkeyscurrentname}%
      \expandafter\edef\csname\string#1path\endcsname{\pgfkeyscurrentpath}%
    \escapechar\count@%
  }
}

\pgfkeys{
  % <path>/.cd scope={<key list>} evaluates <key list> with <path> as
  % default path.
  /handlers/.cd scope/.style={
    \pgfkeyscurrentpath/.get key parts=\myk@key,
    /utils/exec={\expandafter\pgfqkeys\expandafter{\myk@key}{#1}},
  }
}

\pgfkeys{
  % <key>/.export to=<path> similar to <path>/<key-name>/.forward to=<key>
  /handlers/.export to/.style={
    \pgfkeyscurrentpath/.get key parts=\@myk@key,
    #1/\@myk@keyname/.estyle={\@myk@key={##1}},
  }
}

\pgfkeys{
  % <key>/.selected=<value> has the effect of <key>/\pgfkeysvalueof{<key>}=<value>
  % Example:
  %   foo/.initial=a,
  %   foo/.selected=hello, % => foo/a=hello
  %   foo=b,
  %   foo/.selected=world  % => foo/b=world
  /handlers/.selected/.style={
    \pgfkeyscurrentpath/\pgfkeysvalueof{\pgfkeyscurrentpath}={#1},
  }
}

\pgfkeys{
  % <key>/.try selected=<value> has the effect of <key>/\pgfkeysvalueof{<key>}/.try=<value>
  /handlers/.try selected/.style={
    \pgfkeyscurrentpath/\pgfkeysvalueof{\pgfkeyscurrentpath}/.try={#1},
  },
  % '<key>/.is choice with no match' is the same as '<key>/.is choice'
  % with the addition that if '<key>=<value>' is called and '<key>/<value>'
  % is not set, '<key>/.no match=<value>' is called.
  /handlers/.is choice with no match/.style={
    \pgfkeyscurrentpath/.get key parts=\@myk@key,
    \@myk@key/.is choice,
    \@myk@key/.unknown/.ecode={%
      \noexpand\let\noexpand\@unknown@name=\noexpand\pgfkeyscurrentname%
      \noexpand\pgfkeysalso{%
        \@myk@key/.no match/.expand once=\noexpand\@unknown@name,%
      }%
    },
  }
}

\long\def\myk@ifboolexpe#1 then #2 else #3\par{%
  \ifboolexpe{#1}{\pgfkeysalso{#2}}{\pgfkeysalso{#3}}%
}
\long\def\myk@ifboolexpr#1 then #2 else #3\par{%
  \ifboolexpr{#1}{\pgfkeysalso{#2}}{\pgfkeysalso{#3}}%
}

\def\myk@evaluateif#1 then #2 else #3\par{%
  \pgfmathparse{#1}%
  \ifdim\pgfmathresult pt=0pt\relax%
    \pgfkeysalso{#3}%
  \else%
    \pgfkeysalso{#2}%
  \fi%
}

\pgfkeys{
  % /utils/ifboolexpe={<expression>} then {<true key list>}[ else {<false key list>}]
  % Evaluates <expression> using \ifboolexpe and then <true key list>
  % or <false key list>, according to the result.
  /utils/ifboolexpe/.code args={#1 then #2#3}{%
    \ifstrempty{#3}{%
      \myk@ifboolexpe{#1} then {#2} else {}\par%
    }{% else
      \myk@ifboolexpe{#1} then {#2}#3\par%
    }%
  },
  /utils/ifboolexpr/.code args={#1 then #2#3}{%
    \ifstrempty{#3}{%
      \myk@ifboolexpr{#1} then {#2} else {}\par%
    }{% else
      \myk@ifboolexpr{#1} then {#2}#3\par%
    }%
  },
  % /tikz/evaluate if={<expression>} then {<true key list>}[ else {<false key list>}]
  % Evaluates <expression> using \pgfmathparse and then, if the result
  % is different than 0 executes <true key list>, otherwise executes
  % <false key list>.
  /tikz/evaluate if/.code args={#1 then #2#3}{%
    \ifstrempty{#3}{%
      \myk@evaluateif{#1} then {#2} else {}\par%
    }{% else
      \myk@evaluateif{#1} then {#2}#3\par%
    }%
  },
  % For any <key> that was set up as a flag:
  % '<key>/.if true=then {<true key list>}[ else {<false key list>}]'
  % '<key>/.if false=then {<true key list>}[ else {<false key list>}]'
  % evaluate <true key list> or <false key list> according to the flag
  % (i.e. <key>/.@boolexpe).
  /handlers/.if true/.code args={then #1#2}{%
    \edef\myk@key{\pgfkeyscurrentpath}%
    \pgfkeysifdefined{\myk@key/.@boolexpe}{%
      \pgfkeysgetvalue{\myk@key/.@boolexpe}{\myk@boolexpe}%
      \ifstrempty{#2}{%
        \expandafter\myk@ifboolexpe\expandafter{\myk@boolexpe} then {#1} else {}\par%
      }{% else
        \expandafter\myk@ifboolexpe\expandafter{\myk@boolexpe} then {#1}#2\par%
      }%
    }{% else
      \PackageError{pgflibrarymykeys}{'\myk@key' is not a boolean flag}{Did you forget '\myk@key/.is if flag=...'?}%
    }%
  },
  /handlers/.if false/.code args={then #1#2}{
    \edef\myk@key{\pgfkeyscurrentpath}%
    \pgfkeysifdefined{\myk@key/.@boolexpe}{%
      \pgfkeysgetvalue{\myk@key/.@boolexpe}{\myk@boolexpe}%
      \edef\myk@boolexpe{not (\expandonce\myk@boolexpe)}%
      \ifstrempty{#2}{%
        \expandafter\myk@ifboolexpe\expandafter{\myk@boolexpe} then {#1} else {}\par%
      }{% else
        \expandafter\myk@ifboolexpe\expandafter{\myk@boolexpe} then {#1}#2\par%
      }%
    }{% else
      \PackageError{pgflibrarymykeys}{'\myk@key' is not a boolean flag}{Did you forget '\myk@key/.is if flag=...'?}%
    }%
  },
  % <key>/.is flag=<true|false> (default true)
  % Sets up <key> for the '.if true' and '.if false' handlers.
  % <key> or <key>=true to set the flag to true, and <key>=false to set
  % it to false.
  /handlers/.is flag/.style={
    \pgfkeyscurrentpath/.get key parts=\@myk@key,
    \@myk@key/.@boolexpe/.initial=,
    \@myk@key/true/.estyle={\@myk@key/.@boolexpe={bool {true}}},
    \@myk@key/false/.estyle={\@myk@key/.@boolexpe={bool {false}}},
    \@myk@key/.is choice,
    \@myk@key/.default=true,
    \@myk@key={#1},
  },
  % <key>/.is if flag=<TeX-if name>
  % Similar to the PGF '.is if' but also sets up <key> as a flag.
  /handlers/.is if flag/.style={
    \pgfkeyscurrentpath/.get key parts=\myk@key,
    \myk@key/.@boolexpe/.initial={bool {#1}},
    \myk@key/.is if={#1},
  },
  % <key>/.is global if flag=<TeX-if name>
  % Similar to '.is if flag' but sets the 'if' globally.
  /handlers/.is global if flag/.style={
    \pgfkeyscurrentpath/.get key parts=\myk@key,
    \myk@key/.@boolexpe/.initial={bool {#1}},
    \myk@key/true/.code={\global\csname #1true\endcsname},
    \myk@key/false/.code={\global\csname #1false\endcsname},
    \myk@key/.no match/.estyle={
      /errors/boolean expected={\csname myk@key\endcsname}{##1},
    },
    \myk@key/.is choice with no match,
    \myk@key/.default=true,
  },
  % <key>/.is toggle flag=<etoolbox toggle name>
  % Similar to '.is if flag' for flags defined with the etoolbox \newtoggle.
  /handlers/.is toggle flag/.style={
    \pgfkeyscurrentpath/.get key parts=\myk@key,
    \myk@key/.is choice,
    \myk@key/.default=true,
    \myk@key/true/.code={\toggletrue{#1}},
    \myk@key/false/.code={\togglefalse{#1}},
    \myk@key/.@boolexpe/.initial={togl {#1}},
  },
  % <key>/.if defined=then {<true key list>}[ else {<false key list>}]
  % If <key> is defined evaluates <true key list>, otherwise evaluates <false key list>.
  /handlers/.if defined/.style args={then #1#2}{
    \pgfkeyscurrentpath/.get key parts=\myk@key,
    /utils/ifboolexpe={test {\pgfkeysifdefined{\myk@key}}} then {#1}#2,
  },
  % <key>/.if defined value={<macro>} then {<true key list>}[ else {<false key list>}]
  % If <key> is defined, sets <macro> to the value of <key>
  % and evaluates <true key list>, otherwise evaluates <false key list>.
  /handlers/.if defined value/.style args={#1 then #2#3}{
    \pgfkeyscurrentpath/.get key parts=\myk@key,
    \myk@key/.if defined=then {\myk@key/.get=#1,#2}#3,
  },
  % /utils/if no value={<#n>} then {<true keys>}[ else {<false keys>}]
  % evaluates <true keys> if <#n> is \pgfkeysnovalue, otherwise
  % evaluates <false keys>.
  /utils/if no value/.style args={#1 then #2#3}{%
    /utils/exec={\def\myk@temp{#1}},
    /utils/ifboolexpe={test {\ifdefequal{\myk@temp}{\pgfkeysnovalue@text}}} then {#2}#3,
  },
}

\pgfkeys{
  % <key>/.try with warning=<value>
  % Similar to '.try', but when the try fails prints a warning.
  /handlers/.try with warning/.style={
    \pgfkeyscurrentpath/.get key parts=\myk@key,
    \myk@key/.try={#1},
    /utils/ifboolexpe={not bool {pgfkeyssuccess}} then {%
      /utils/package warning/.expanded={pgflibrarymykeys}{The key '\myk@key' does not exist},
    },
  },
}

\pgfkeys{
  % <key>/.get with default={<macro>}{<value>}
  % Similar to /.get, but if <key> is not defined assignes <value> to
  % <macro> using \def.
  /handlers/.get with default/.style 2 args={
    \pgfkeyscurrentpath/.get key parts=\myk@key,
    \myk@key/.if defined=then {
      \myk@key/.get=#1,
    } else {
      /utils/exec={\def#1{#2}},
    },
  },
  % <key>/.value of=<another key>
  % Equivalent to writing '<key>=\pgfkeysvalueof{<another key>}'
  % which is somewhat like the '.link' handler, only for predefined keys.
  % NOTE: <another key> must be a fully qualified key name.
  /handlers/.value of/.style={\pgfkeyscurrentpath=\pgfkeysvalueof{#1}},
}

\pgfkeys{
  % <key>/.global store in=<macro>
  % Similar to '.store in', except that <macro> is defined globally.
  /handlers/.global store in/.style={
    \pgfkeyscurrentpath/.code={\gdef#1{##1}},
  }
}

\pgfkeys{
  % <key>/.list append={<value>}
  % <key>/.list prepend={<value>}
  % <key>/.list reverse
  % Assuming the value of <key> is a comma separated list, apply the
  % selected list operation to it.
  /handlers/.list append/.style={
    \pgfkeyscurrentpath/.get key parts=\myk@key,
    \myk@key/.get=\myk@temp@the@list,
    /utils/ifboolexpe={test {\ifdefempty{\myk@temp@the@list}}} then {
      \myk@key={#1}
    } else {
      \myk@key/.append={,#1}
    },
  },
  /handlers/.list prepend/.style={
    \pgfkeyscurrentpath/.get key parts=\myk@key,
    \myk@key/.get=\myk@temp@the@list,
    /utils/ifboolexpe={test {\ifdefempty{\myk@temp@the@list}}} then {
      \myk@key={#1}
    } else {
      \myk@key/.prefix={#1,}
    },
  },
  /handlers/.list reverse/.style={
    \pgfkeyscurrentpath/.get key parts=\myk@key,
    \myk@key/.get=\myk@temp@the@list,
    \myk@key=,
    \myk@key/.list prepend/.list/.expanded={\myk@temp@the@list},
  }
}

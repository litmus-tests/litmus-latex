\documentclass[a4paper]{article}
\usepackage[utf8]{inputenc}
\usepackage[T1]{fontenc}
\usepackage[scaled=0.82]{beramono}
\usepackage[margin=2cm]{geometry}

\setlength\parindent{0pt}

%%%%%%%%%%%%%%%%%%%%%%%%%%%%%%%%%%%%%%%%%%%%%%%%%%%%%%%%%%%%%%%%%%%%%%

\def\myautoref#1#2#3{\autoref{#1}}

\usepackage[index,hyperref]{litmus}
\litmuscfg{%
  % Externalise the litmus figures
%   extern=false,
%   strict translation=false,
  % The default value of the optional argument of \litmus (i.e. the
  % source arichetecture)
%   default arch=GEN,
  % The target architecture of litmus names of \litmus (empty means
  % same as the source arch)
%   target arch=,
  % The target architectures of litmus figures of \litmus (empty means
  % same as the source arch)
  target figs={A64,RV,PPC},
  % Typeset \litmusref
%   litmus ref=...,% \ref{#1} and gobble the following two arguments
%   context=main,
  % Draw a midrule in experimental results tables every 5 lines
  exp results midrule=5,
}

\tikzset{
  % Figures use sans serif
  node font=\sffamily,
  % Nice arrow heads
  >=litmus arrow,
}

\litmusset{
  every litmus/.append style={
    every thread header/.append style={node font=\small\sffamily},
  },
  every litmus table/.append style={
    every state/.append style={state prefix font=\small\sffamily},
  },
}

\newcommand{\TikZ}{Ti\textit{k}Z}

%%%%%%%%%%%%%%%%%%%%%%%%%%%%%%%%%%%%%%%%%%%%%%%%%%%%%%%%%%%%%%%%%%%%%%

\usepackage{hyperref}
\hypersetup{
    linkbordercolor={1 1 1},
    colorlinks=true,
    linkcolor=violet,
%     citebordercolor={0.5 1 0.5},
}

%%%%%%%%%%%%%%%%%%%%%%%%%%%%%%%%%%%%%%%%%%%%%%%%%%%%%%%%%%%%%%%%%%%%%%

\begin{document}

\section{Macros}
\verb|\litmus{<name>}| - adds a figure of the litmus, if one does not
already exist, typesets the litmus name and hyperlinks it to the
figure, like this: \litmus{MP}.

\verb|\litmus+{<name>}| - similar to \verb|\litmus{<name>}| except
that it forces a new figure, even if a previous one already exists.

\verb|\litmus-{<name>}| - similar to \verb|\litmus{<name>}| except
that it never adds a figure (for forward referencing).

\verb|\litmus*{<name>}| - just typesets the litmus name (without a
figure or a link), like this: \litmus*{MP}.

\vspace{4ex}

\verb|\litmusref{<name>}| - reference a litmus figure, like this: \litmusref{MP} (the \litmus{MP} litmus figure).

\vspace{4ex}

\section{litmuscfg}
The followings are options that can be changed by
\verb|\litmuscfg{<options>}|.

\verb|index| - can only be used in the preamble. Loads the index
package and generates an index of litmus tests (use
\verb|\printindex[litmus]| to typeset the index).

\verb|hyperref| - can only be used in the preamble. Loads the hyperref
package and links litmus names to their figures.

\verb!extern=false|true! - (initially false, default true) can only be
used in the preamble. Externalise the litmus figures, that is,
generates a separate pdf for each diagram (once), and then include
those pdf in the document instead of typesetting them every time the
document is built.

\verb!strict translation=false|true! - (initially false, default true)
be strict when translating litmus name between architectures (strict
will fail if an edge has no mapping, non-strict will use the source
name of the edge).

\verb|default arch=<arch>| - (initially GEN) the default value of the
optional argument of \verb|\litmus|.
For example, \verb|\litmus*{MP+fens}| parses the litmus name as a
generic (GEN) name and \verb|\litmus*[A64]{MP+dmb.sys}| parses the
litmus name as an AArch64 name.

\verb|target arch=<arch>| - (initially empty) the target architecture
of litmus names of \verb|\litmus| (empty means same as the source arch).
For example
\verb|\litmuscfg{target arch=PPC}\litmus*[A64]{MP+dmb.sys}|
will translate the AArch64 name \litmus*[A64]{MP+dmb.sys} to the
Power name \litmus*[PPC]{MP+syncs}.

\verb|target figs=<archs>| - (initially empty) a CSV of the architectures
that are to appear in litmus diagrams (empty means same as the source arch).

\verb|litmus ref=<macro>| - (initially equivalent to \verb|\ref{#1}|)
\verb|<macor>| should take three arguments and typeset a reference to
a litmus figure. The arguments are: \#1 - label of the litmus figure;
\#2 - target architecture; \#3 - target name.

\verb|context=<name>| - (initially main) by changing context you can
include multiple figures of the same litmus test and reference each
one specifically.

\verb!exp results midrule=none|<n>! - (initially none) determines if a
midrule should be drawn every \verb|<n>| lines in experimental results
tables.
With no argument the option is enabled with the previous value of
\verb|<n>| (initially 5).

\section{litmus.names.cfg}
This file defines how litmus names are translated between architectures.

The translation does not change the basic shape name (e.g.\ MP), which
is the prefix of the name made of anything that is not a lower case
letter, but not including the last `+' if the prefix ends with one
(e.g.\ the name of 2+2W+dmb.sys is 2+2W).

The suffix of the name is then split to edges, separated by `+' and `-',
And each edge is translated according to the mapping.

\verb|\litmusedgetrans{<src>}{<e>}{<trg1>:<e1>,...}| - defines an
edge mapping from the source architecture \verb|<src>| edge \verb|<e>|
to the target architecture \verb|<trgn>| edge \verb|<en>| for every
\verb|n| in the list.
For example \verb|\litmusedgetrans{GEN}{fen}{A64:dmb.sy,PPC:sync}|.

\verb|\litmusedgetransannot{<src>/<trg>}{<s1>/<t1>,...}| - defines
a mapping for edge annotations from the source architecture \verb|<src>|
to the target architecture \verb|<trg>|.
For example \verb|\litmusedgetransannot{GEN/A64}{p/p,aq/a,rl/l}|.

\verb|\newlitmusnametrans{<src>}{<name>}{<trg>}{<name'>}| - defines a
special case for the architecture \verb|<src>| litmus \verb|<name>|
and target architecture \verb|<trg>|.
For example: \verb|\newlitmusnametrans{GEN}{MP+addrs}{A64}{MP+addrs+V2}|.

\verb|\renewlitmusnametrans{<src>}{<name>}{<trg>}{<name'>}| - similar to
\verb|\newlitmusnametrans{<src>}{<name>}{<trg>}{<name'>}|.


\cleardoublepage
% \phantomsection
% \addcontentsline{toc}{chapter}{Index of Litmus Tests}
\printindex[litmus][Numbers written in bold refer to the page where a
figure of the litmus test is displayed.]

\end{document}

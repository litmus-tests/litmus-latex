%%%%%%%%%%%%%%%%%%%%%%%%%%%%%%%%%%%%%%%%%%%%%%%%%%%%%%%%%%%%%%%%%%%%%%
%                                                                    %
%         rmem executable model                                      %
%                                                                    %
%   Shaked Flur, University of Cambridge                             %
%   Jon French, University of Cambridge                              %
%   Kathryn Gray, University of Cambridge (when this work was done)  %
%   Robert Norton-Wright, University of Cambridge                    %
%   Christopher Pulte, University of Cambridge                       %
%   Susmit Sarkar, University of St Andrews                          %
%   Peter Sewell, University of Cambridge                            %
%                                                                    %
%  This file is copyright 2010-2017 by the authors above.            %
%                                                                    %
%  All rights reserved.                                              %
%%%%%%%%%%%%%%%%%%%%%%%%%%%%%%%%%%%%%%%%%%%%%%%%%%%%%%%%%%%%%%%%%%%%%%

% This library should be loaded using '\usetikzlibrary{litmus}'
% See tikz_example/example.tex for details on how to use this library

%%% TODO:
% - Add @ to all the keys that should not be accessible to users.
% - Evaluating multiple styles in a ".cd scope" is not safe, as the
%   first style might change the default path.

%%%%%%%%%%%%%%%%%%%%%%%%%%%%%%%%%%%%%%%%%%%%%%%%%%%%%%%%%%%%%%%%%%%%%%

\RequirePackage{etoolbox}

% Load pgf/tikz libraries
\usepgflibrary{mykeys}
\usetikzlibrary{arrows.meta}
\usetikzlibrary{positioning}
\usetikzlibrary{calc}
\usetikzlibrary{chains}
\usetikzlibrary{scopes}
\usetikzlibrary{matrix}
\usetikzlibrary{fit}
\usetikzlibrary{backgrounds}
\usetikzlibrary{math}

% \lit@ifnodeexists{<name or alias>}{<true>}{<false>}
\newcommand{\lit@ifnodeexists}[1]{%
  \ifcsequal{pgf@sh@pi@#1}{pgfpictureid}%
}

%%%%%%%%%%%%%%%%%%%%%%%%%%%%%%%%%%%%%%%%%%%%%%%%%%%%%%%%%%%%%%%%%%%%%%

\pgfkeys{
  /litmus/.is family,
  %%% pgf BUG: when defining "<path>/.search also=..." and then calling
  %%% "\pgfkeys{<path>/<unknown key>}" the error message complains that
  %%% "/handlers/.unknown/.@cmd" does not exist, instead of complaining
  %%% about "<path>/<unknown key>", this makes debugging harder.
  /litmus/.search also={/tikz},
  %
  % This is a bit of a hack to allow un-prefixed /litmus keys when
  % /tikz is the default path.
  % NOTE: /tikz/.unknown/.@cmd calls '/errors/unknown key' after
  % trying everything and failing.
  /litmus/append to tikz path search/.style={
    % Setup things only if we have not already done it
    /errors/unknown key/original/.@cmd/.if defined=then {} else {
      % Save a copy of the original '/errors/unknown key'
      /utils/exec={%
        \pgfkeysgetvalue{/errors/unknown key/.@cmd}{\lit@temp@unknown}%
        \pgfkeyslet{/errors/unknown key/original/.@cmd}{\lit@temp@unknown}%
      },
      % Redefine '/errors/unknown key' to look in /litmus before failing
      /errors/unknown key/.style 2 args={
        % ##1 is the unknown key (fully qualified)
        % ##2 is the value passed to the key
        ##1/.get key parts=\lit@key,
        /utils/ifboolexpr={test {\ifdefstring{\lit@keypath}{/tikz}}} then {
          % /tikz/.unknown uses \pgfkeys (instead of \pgfkeysalso) to
          % call "/errors/unknown key" so we need to switch back to /tikz
          /tikz,
          /litmus/\lit@keyname/.try={##2},
          /errors/unknown key/original/.retry={##1}{##2},
        } else {
          /errors/unknown key/original={##1}{##2},
        },
      },
      % By forwarding the keys "/errors/unknown key/.code 2 args" and
      % "/errors/unknown key/.style 2 args" we make it harder for
      % someone to overwrite the code above (instead the saved copy
      % will be overwritten).
      /errors/unknown key/.code 2 args/.forward to=/errors/unknown key/original/.code 2 args,
      /errors/unknown key/.style 2 args/.forward to=/errors/unknown key/original/.style 2 args,
    },
  },
}
\def\litmusset{\pgfqkeys{/litmus}}

\def\assem{\verb}

% By default \assem is \verb which cannot be used in macro arguments.
% Whenever we need to pass \assem in an argument we pass this instead:
\def\lit@internal@assem#1{{\ttfamily #1}}

% \chaincount{<chain>} returns the number of nodes on the chain
\newcommand{\chaincount}[1]{\csuse{tikz@lib@chain@count@#1}}

\newif\iflit@assem\lit@assemtrue
\litmusset{is assem litmus/.is if flag=lit@assem}

\newif\iflit@events\lit@eventstrue
\litmusset{is events litmus/.is if flag=lit@events}

\newif\iflit@hide@line\lit@hide@linefalse
\litmusset{hide line/.is global if flag=lit@hide@line}

\newif\iflit@needs@end@row\lit@needs@end@rowfalse

\newcounter{lit@ioid@sub}

\newif\iflit@last@inst\lit@last@instfalse

\def\lit@add@row@after#1#2{%
  % #1 - ioid of the instruction above
  % #2 - node text
  \iflit@assem%
    \lit@matrixendrow%
    \stepcounter{lit@ioid@sub}%
    \node[/litmus/assem/.expanded={#1-\arabic{lit@ioid@sub}}] {#2};%
  \fi%
}

% This is the end-of-row macro for 'instructions' matrix.
\def\litmusendinst{\futurelet\lit@tok\lit@endinst}
\def\lit@endinst{%
  \ifx\lit@tok\egroup%
    \global\lit@last@insttrue%
  \else%
    \global\lit@last@instfalse%
  \fi%
  \iflit@assem%
    \iflit@hide@line%
      \iflit@last@inst%
        \PackageWarning{litmus-tikz-library}{last line can not be completely hidden (\lit@ioid)}%
      \fi%
    \else%
      \pgfkeysifdefined{/tikz/pics/\lit@ioid/.@cmd}{%
        \iflit@events%
          \pgfmatrixnextcell%
          \pic {\lit@ioid};%
        \fi%
      }{}%
      % add rows as requested
      \pgfkeysifdefined{/litmus/assem \lit@ioid/insert rows after/.@cmd}{%
        \setcounter{lit@ioid@sub}{0}%
        \litmusset{assem \lit@ioid/insert rows after}%
      }{}%
      \iflit@last@inst\else%
        \lit@matrixendrow%
      \fi%
    \fi%
  \else%
    \pgfkeysifdefined{/tikz/pics/\lit@ioid/.@cmd}{%
      % The matrix must end with \pgfmatrixendrow (the one at the end
      % of this macro) so we have to do all the other \pgfmatrixendrow
      % just before the \pic, except, of course, for the very first \pic
      \iflit@needs@end@row%
        \lit@matrixendrow%
      \fi%
      \lit@needs@end@rowtrue%
      \pic {\lit@ioid};%
    }{}%
  \fi%
  \iflit@last@inst%
    \expandafter\lit@matrixendrow%
  \fi%
}

\def\lit@matrixendrow{\expandafter\pgfmatrixendrow\expandafter[\lit@add@space@after@row]}

%%%%%%%%%%%%%%%%%%%%%%%%%%%%%%%%%%%%%%%%%%%%%%%%%%%%%%%%%%%%%%%%%%%%%%
% typeset memory footprint

\newif\iflit@mixed@size\lit@mixed@sizetrue
\litmusset{is mixed-size/.is if flag=lit@mixed@size}

\litmusset{
  non-mixed-size/.style={
    /litmus/memfp/.code n args={3}{%
      \tikzmath{
        let \addr = ##1;
        let \offs = ##2;
        let \size = ##3;
        if \offs==0 then { print{\addr}; }
        else { print{\addr+\offs}; };
      }%
    },
    /litmus/memrf/.code 2 args=rf,
    /litmus/memfr/.code 2 args=fr,
    /litmus/is mixed-size=false,
  },
}

\litmusset{
  old-mixed-size/memslices/.code 2 args={%
    \def\slices@temp{\@gobble}%
    \foreach \i/\j in {#2} {%
      \tikzmath{\size = int(\j - \i + 1);}%
      \ifnum\i=\z@%
        \xdef\slices@temp{\slices@temp,#1/\size}%
      \else%
        \xdef\slices@temp{\slices@temp,#1+\i/\size}%
      \fi%
    }%
    \slices@temp%
  },
  old-mixed-size/.style={
    /litmus/memfp/.code n args={3}{%
      \tikzmath{
        let \addr = ##1;
        let \offs = ##2;
        let \size = ##3;
        if \offs==0 then { print{\addr/\size}; }
        else { print{\addr+\offs/\size}; };
      }%
    },
    /litmus/memrf/.code 2 args={rf\,\pgfkeysalso{/litmus/old-mixed-size/memslices={##1}{##2}}},
    /litmus/memfr/.code 2 args={fr\,\pgfkeysalso{/litmus/old-mixed-size/memslices={##1}{##2}}},
    /litmus/is mixed-size=true,
  },
}

\litmusset{
  mixed-size/memslices/.code 2 args={%
    \def\slices@temp{\@gobble}%
    \foreach \i/\j in {#2} {%
      \ifx\i\j%
        \xdef\slices@temp{\slices@temp,\i}%
      \else%
        \xdef\slices@temp{\slices@temp,\i..\j}%
      \fi%
    }%
    #1[\slices@temp]%
  },
  mixed-size/.style={
    /litmus/memfp/.code n args={3}{%
      \tikzmath{
        let \addr = ##1;
        let \offs = ##2;
        let \size = ##3;
        \i = int(\offs + \size - 1);
        if \size==1 then { print{\addr[\offs]}; }
        else { print{\addr[\offs..\i]}; };
      }%
    },
    /litmus/memrf/.code 2 args={rf\,\pgfkeysalso{/litmus/mixed-size/memslices={##1}{##2}}},
    /litmus/memfr/.code 2 args={fr\,\pgfkeysalso{/litmus/mixed-size/memslices={##1}{##2}}},
    /litmus/is mixed-size=true,
  },
}

\litmusset{non-mixed-size}

%%%%%%%%%%%%%%%%%%%%%%%%%%%%%%%%%%%%%%%%%%%%%%%%%%%%%%%%%%%%%%%%%%%%%%

\newcount\maxassemlines

\newcommand{\lit@check@exists}{%
  {%
    \def\lit@check@exists@do##1{%
      \ifinlist{##1}{\lit@exists@list}{}{% else
        \PackageError{litmus-tikz-library}{event '##1' which you tried to change does not exist}{}%
      }%
    }%
    \forlistloop{\lit@check@exists@do}{\lit@check@exists@list}%
  }%
}

\litmusset{
  litmus arrow/.tip={Stealth[length=0.5em,inset=0.1em]},
  litmus internal/.style={
    /utils/exec={\global\maxassemlines=0},
    % If the assem/event 'text height' of the first instructions are
    % not identical the thread headers will not be aligned
    /tikz/start chain=threads going {/tikz/right=of \tikzchainprevious.north east, /tikz/matrix anchor=north west},
    /tikz/>=litmus arrow,
    /utils/exec={
      \gdef\lit@exists@list{}
      \gdef\lit@check@exists@list{}
    },
    /tikz/execute at end scope=\lit@check@exists,
  },
  kind/.is choice,
  kind/.default=full,
  kind/full/.style={
    /litmus/is assem litmus=true,
    /litmus/is events litmus=true,
    /litmus/simple instructions/.style={/litmus/@simple full},
    /litmus/litmus internal,
    /litmus/threads distance/.initial=3em,
    /litmus/header to instructions distance/.initial=0pt,
    /litmus/draw thread header/.style=,
    /litmus/disable edge routing,
    /litmus/.cd scope={
      every litmus/.try,
      every full litmus/.try,
    },
  },
  kind/assem/.style={
    /litmus/is assem litmus=true,
    /litmus/is events litmus=false,
    /litmus/simple instructions/.style={/litmus/@simple assem},
    /litmus/litmus internal,
    /litmus/threads distance/.initial=0pt,
    /litmus/header to instructions distance/.initial=0pt,
    /litmus/draw thread header/.style=,
    /litmus/disable edge routing,
    /litmus/.cd scope={
      every litmus/.try,
      every assem litmus/.try,
    },
%     /litmus/event/.append style=/litmus/hide,
%     /litmus/events box/.style=/litmus/hide,
    /litmus/instruction relations/.style=/litmus/hide,
    /litmus/event relations/.style=/litmus/hide,
  },
  kind/events/.style={
    /litmus/is assem litmus=false,
    /litmus/is events litmus=true,
    /litmus/simple instructions/.style={/litmus/@simple events},
    /litmus/litmus internal,
    /litmus/threads distance/.initial=3em,
    /litmus/header to instructions distance/.initial=0.5ex,
    /litmus/draw thread header/.style={/tikz/rectangle, /tikz/draw},
    /litmus/enable edge routing,
    /litmus/.cd scope={
      every litmus/.try,
      every events litmus/.try,
    },
    /litmus/code label position=none, % don't show code labels...
    % ...and change the key so it cannot be changed (e.g. by 'every litmus')
    /litmus/code label position/.code=,
    /litmus/assem internal/.append style=/litmus/hide,
  },
}

% This is a hack for .ps as it does not support opacity:
\@ifclasswith{standalone}{dvips}{
  \litmusset{
    kind/events/.append style={
      /litmus/assem internal/.append style={/tikz/scale=0.001, /tikz/transform shape},
    },
  }
}{}

%%%%%%%%%%%%%%%%%%%%%%%%%%%%%%%%%%%%%%%%%%%%%%%%%%%%%%%%%%%%%%%%%%%%%%

\tikzset{
  % The alias key does not work on matrices, that breaks 'on chain',
  % we fix it by adding another node with the expected name
  matrix on chain/.style={
    /tikz/matrix,
    /tikz/on chain={#1},
    % We forward all the alias arguments to 'matrix alias' and use it
    % to collect the aliases in \lit@alias@temp (csv)
    /utils/exec={\def\lit@alias@temp{\@gobble}},
    /temp/matrix alias/.code={\xdef\lit@alias@temp{\lit@alias@temp,##1}},
    /tikz/alias/.forward to=/temp/matrix alias,
    % We do not want to collect aliases in the cells
    /tikz/execute at begin cell={\pgfkeys{/temp/matrix alias/.code={}}},
    % Add to the end of the path a new simple node with all the aliases
    /tikz/append after command={
      node[fit=(\tikzlastnode),
        minimum size=0, inner sep=0, outer sep=0,
        alias/.list/.expand once=\lit@alias@temp,
      ] {}
    },
  },
}

% Hold the typical height and depth of assem text
\newlength{\assemheight}
\newlength{\assemdepth}

\litmusset{
  add thread header/.is flag=false,
  thread/.style={
    /litmus/thread id/.store in=\lit@tid,
    /litmus/thread id={#1},
    /litmus/add thread header,
    /litmus/.cd scope={
      every thread/.try=#1,
      thread #1/.try,
    },
  },
  thread header text/.store in=\lit@thread@header@text,
  thread header text={Thread~\lit@tid},
  thread header/.style={
    /tikz/alias/.expanded={thread \lit@tid\space header},
    /litmus/draw thread header/.try,
    /tikz/node distance/.value of=/litmus/header to instructions distance,
    /litmus/.cd scope={
      every thread header/.try/.expand once=\lit@tid,
      thread \lit@tid\space header/.try,
    },
  },
  ioid/.global store in=\lit@ioid,
  the code label/.global store in=\lit@code@label,
  instructions/.style={
    /utils/if no value={#1} then {} else {
      % There is no thread scope, but we still need to set \lit@tid.
      % We use global because '/tikz/append after command' is executed
      % in a different group.
      /litmus/thread id/.global store in=\lit@tid,
      /litmus/thread id={#1},
    },
    /litmus/the code label=,
    % 'node distance' must be set before 'on chain'
    /tikz/node distance/.value of=/litmus/threads distance,
    /tikz/matrix on chain=threads,
    /tikz/column sep=0.5em,
    /utils/exec={
      % Set \assemheight and \assemdepth to reasonable values
      \settoheight{\assemheight}{\pgfinterruptpicture\lit@internal@assem{X[/}\endpgfinterruptpicture}
      \global\assemheight=\assemheight
      \settodepth{\assemdepth}{\pgfinterruptpicture\lit@internal@assem{gjpqyQ[,/}\endpgfinterruptpicture}
      \global\assemdepth=\assemdepth
    },
    /litmus/is assem litmus/.if true=then {
      /litmus/is events litmus/.if true=then {
        /tikz/row sep={3ex,between origins}
      } else {
        /tikz/row sep={\assemheight+\assemdepth+1pt,between origins}
      }
    } else {
      /tikz/row sep={7ex,between origins}
    },
    /tikz/inner sep=0,
    /tikz/outer sep=0,
    /litmus/.cd scope={
      every instructions/.try,
    },
    /utils/exec={
      % Set \assemrowsep to the 'row sep' length part:
      \begingroup
        \def\@tempA##1,##2\par{##1}
        \edef\@tempB{\pgfmatrixrowsep}
        \edef\@tempC{\endgroup\noexpand\xdef\noexpand\assemrowsep{\expandafter\@tempA\@tempB,\par}}
      \@tempC
    },
    /litmus/add space after row/.store in=\lit@add@space@after@row,
    /litmus/add space after row=0pt,
    /tikz/execute at end matrix={%
      \ifnum\pgfmatrixcurrentrow>\maxassemlines\relax%
        \global\maxassemlines=\pgfmatrixcurrentrow%
      \fi%
    },
    /litmus/add thread header/.if true=then {
      /tikz/append after command={
        node[%
          % We check if \tikz@node@at has changed (becuase of user defined
          % styles), and if it has not, we set it to be above the instructions
          /utils/exec={\let\lit@temp@node@at=\tikz@node@at},
          /litmus/thread header,
          /utils/ifboolexpr={%
            test {\ifdefequal{\tikz@node@at}{\lit@temp@node@at}}%
          } then {above=of \tikzlastnode},
        ] {\lit@thread@header@text}%
      },
    },
  },
}

\newif\iflit@simple@endcell\lit@simple@endcelltrue
\def\lit@simple@event@endcell{\iflit@simple@endcell\egroup;\fi}

\def\lit@simple@cell#1{%
  \@ifnextchar !{%
    \lit@simple@endcellfalse\@gobble%
  }{% else
    \lit@simple@endcelltrue #1%
  }%
}
\def\lit@simple@event#1:{%
  \node[/litmus/.cd scope={event={#1}}] \bgroup%
    % when using "text width" tikz wraps the node text with a minipage that
    % redefines \\, we redefine it again so that "/tikz/execute at end cell"
    % can run as expected
    \let\\=\pgfmatrixendrow%
}


\def\lit@simple@assem{%
  \node[/litmus/assem] \bgroup%
    % when using "text width" tikz wraps the node text with a minipage that
    % redefines \\, we redefine it again so that "/tikz/execute at end cell"
    % can run as expected
    \let\\=\pgfmatrixendrow%
    \assem%
}

% a few wrappers of 'instructions' that make it easier to type simple tests
\litmusset{
  events column/.style={ % e.g.: a:R x=0
    /tikz/execute at begin cell={\lit@simple@cell\lit@simple@event},
    /tikz/execute at end cell={\expandafter\lit@simple@event@endcell},
  },
  assem column/.style={ % e.g.: |LDR W1,[X2]|
    /tikz/execute at begin cell={\lit@simple@cell\lit@simple@assem},
    /tikz/execute at end cell={\expandafter\lit@simple@event@endcell},
  },
  @simple instructions/.style={
    /litmus/add thread header,
    /litmus/instructions/.expanded=\chaincount{threads},
    /litmus/.cd scope={
      every simple instructions/.try,
    },
  },
  @simple events/.style={
    /litmus/@simple instructions,
    /tikz/column 1/.style=/litmus/events column,
    /litmus/.cd scope={
      every simple events/.try,
    },
  },
  @simple assem/.style={
    /litmus/@simple instructions,
    /tikz/column 1/.style=/litmus/assem column,
    /litmus/.cd scope={
      every simple assem/.try,
    },
  },
  @simple full/.style={
    /litmus/@simple instructions,
    /tikz/column 1/.style=/litmus/assem column,
    /tikz/column 2/.style=/litmus/events column,
    /litmus/.cd scope={
      every simple full/.try,
    },
  },
}

\newcommand{\lit@auxout@comtextwidth}[2]{%
  \immediate\write\@auxout{\string\lit@load@width{#1}{#2}}
}

\newcommand{\lit@load@width}[2]{%
  \csgdef{#1@litmusinstructionswidths}{#2}%
  \foreach \w [count=\t from 0] in {#2}{%
    \csxdef{#1@litmusinstructionswidth@\t}{\w}%
  }%
}

\pgfmathdeclarefunction{computedwidthofthread}{2}{%
  \begingroup%
    \ifcsdef{#1@litmusinstructionswidth@#2}{%
      \edef\res{\csuse{#1@litmusinstructionswidth@#2}}%
      \expandafter\pgf@x\res\relax%
    }{% else
      %%% NOTE: ^^J is a line-break, litmus.mk is looking for '^run latex again.'
      \PackageWarning{litmus-tikz-library}{Computed assem width of #1 is not available,^^Jrun latex again.}%
      \pgf@x0pt\relax%
    }%
    \pgfmathreturn\pgf@x%
  \endgroup%
}
\pgfkeys{
  /pgf/declare function={
    computedwidthof(\@uid)=computedwidthofthread("\@uid",\lit@tid);
  },
}

\colorlet{code label color}{black!50}

\litmusset{
  add row after/.style 2 args={%
    /litmus/assem #1/insert rows after/.append code={\lit@add@row@after{#1}{#2}}%
  },
  collect eiids/.style={ % #1 - an ioid
    /litmus/#1/eiids/.if defined value={\lit@eiids} then {
      % Barriers have no \pic but we still want an eiid comment
      /litmus/assem #1/eiids/.initial/.expanded=\lit@eiids,
    } else {
      /tikz/#1/eiids/.if defined value={\lit@eiids} then {
        /litmus/assem #1/eiids/.initial/.expanded=\lit@eiids,
      } else {
        /tikz/pics/#1-eiids/.@cmd/.if defined=then {
          /utils/exec={
            % Create the ioid's \pic without drawing it, to get the list
            % of eiids (placed in \lit@eiids).
            \begingroup
              \pgfpositionnodelater{\empty} % When a node is created, it is not
                                            % inserted into the current picture.
                                            % see 101.2.3 Deferred Node Positioning
              \pic {#1-eiids};
            \endgroup
          },
          /litmus/assem #1/eiids/.initial/.expand once=\lit@eiids,
        },
      },
    },
  },
  compute assem text width/.value required,
  compute assem text width/.style={
    % #1: some unique identifier; Used internally to store the widths.
    %     The pgf math function 'computedwidthof("#1")' expands to the
    %     computed widths of this diagram.
    /utils/ifboolexpe={test {\ifcsdef{#1@litmusinstructionswidths@uid}}} then {
      /utils/package warning={litmus-tikz-library}{'compute assem text width' expects a unique identifier, '#1' was already used.},
    } else {
      /utils/exec={\csgdef{#1@litmusinstructionswidths@uid}{true}},
      /pgf/declare function={
        computedwidth()=computedwidthof("#1");
      },
      /utils/ifboolexpe={test {\ifcsdef{#1@litmusinstructionswidths}}} then {
        % Width has been previously computed, saved to the .aux file and
        % loaded. Set 'text width' if it has not been set manually.
        /utils/exec={\lit@auxout@comtextwidth{#1}{\csuse{#1@litmusinstructionswidths}}},
        /litmus/computed text width/.style={
          /utils/ifboolexpe={test {\ifdefempty{\tikz@text@width}}} then {
            /tikz/text width=computedwidth,
          },
        },
      } else {
        % Width has not been computed in a previous run, compute it now
        % and save to the .aux file
        %%% NOTE: ^^J is a line-break, litmus.mk is looking for '^run latex again.'
        /utils/package warning={litmus-tikz-library}{Computing width of assem (#1),^^Jrun latex again.},
        /litmus/computed text width/.style={
          /tikz/text width=,
          /tikz/minimum size=0,
          /tikz/inner sep=0,
          /tikz/outer sep=0,
          /tikz/draw=none,
        },
        /tikz/execute at end scope={
          \def\lit@temp@widths{\@gobble}
          \tikzmath{
            int \t;
            coordinate \d;
            for \t in {1,...,int(\chaincount{threads})}{
              \d = (threads-\t.east)-(threads-\t.west);
              print{\xdef\lit@temp@widths{\lit@temp@widths,\dx}};
            };
          }
          \lit@auxout@comtextwidth{#1}{\lit@temp@widths}
        },
      },
    },
  },
  assem internal/.style={
    % #1 - ioid
    /tikz/name=assem #1,
    /litmus/ioid={#1},
    /litmus/hide line=false,
    /tikz/anchor=base west,
    % This makes assem rows consistent (i.e. rows of different threads align with each other)
    /tikz/text depth=\assemdepth,
    /tikz/text height=\assemrowsep - \assemdepth,
    /litmus/.cd scope={
      every assem/.try={\lit@code@label}{#1},
      assem #1/.try={\lit@code@label}{#1},
    },
    % 'execute at begin/end node' are accumelating so these have to be
    % after the user styles so the user's code is in the real begining
    % and user's end is before the eiids comment.
    /tikz/execute at begin node={
      \litmusset{inline code label/.selected/.expanded={\lit@code@label}{#1}}
    },
    /tikz/execute at end node={
      \litmusset{show eiids comment/.selected={#1}}
    },
    /litmus/computed text width/.try,
  },
  % 0-0 is a dummy ioid for hand written tests
  assem/.default=0-0,
  assem/.style={
    % #1 - ioid
    /litmus/collect eiids={#1},
    /litmus/assem internal={#1},
    % We always draw the edges, even when the events are not drawn,
    % hence, we add aliases to the assem so it can take the events place.
    /litmus/is events litmus/.if false=then {
      /litmus/assem #1/eiids/.if defined value={\lit@eiids@temp} then {
        /tikz/alias={#1},
        /tikz/alias/.list/.expand once=\lit@eiids@temp,
      },
    },
    /tikz/execute at end node={\gdef\lit@code@label{}},
  },
}

\litmusset{
  code label/.code 2 args={%
    % #1 - location label
    % #2 - ioid
    \ifstrempty{#1}{}{% else
      {\color{code label color}\lit@internal@assem{#1:}}%
    }%
  },
  inline code label/.initial=none,
  inline code label/inline/.style 2 args={/litmus/.cd scope={code label={#1}{#2}}},
  inline code label/left/.code 2 args={\llap{\litmusset{code label={#1}{#2}}}},
  inline code label/none/.style 2 args=,
  code label position/.is choice,
  code label position/above/.style={
    /litmus/assem label/.style={
      /litmus/the code label={##1},
      /tikz/node contents={%
        \litmusset{code label/.expanded={\lit@code@label}{\lit@tid-##1}}%
      },
      /litmus/assem internal/.expanded={\lit@tid-##1},
      /litmus/.cd scope={
        every assem label/.try,
      },
    },
    /litmus/inline code label=none,
  },
  code label position/inline/.style={
    /litmus/assem label/.style={
      /litmus/the code label={##1},
      /tikz/node contents=,
      /litmus/hide,
      /litmus/hide line,
    },
    /litmus/inline code label=inline,
  },
  code label position/left/.style={
    /litmus/assem label/.style={
      /litmus/the code label={##1},
      /tikz/node contents=,
      /litmus/hide,
      /litmus/hide line,
    },
    /litmus/inline code label=left,
  },
  code label position/none/.style={
    /litmus/assem label/.style={
      /litmus/ioid={##1},
      /litmus/the code label={##1},
      /tikz/node contents=,
      /litmus/hide,
      /litmus/hide line,
    },
    /litmus/inline code label=none,
  },
  code label position=above,
  eiids comment/.code={%
    \hfill{\color{red!60}\lit@internal@assem{//#1}}%
  },
  show eiids comment/.initial={\iflit@events false\else true\fi},
  show eiids comment/.default=true,
  show eiids comment/true/.style={
    /litmus/assem #1/eiids/.if defined value={\lit@eiids@temp} then {
      /litmus/.cd scope={
        eiids comment #1/.try/.expanded=\lit@eiids@temp,
        eiids comment/.retry/.expanded=\lit@eiids@temp,
      },
    },
  },
  show eiids comment/false/.style={},
}

\pgfkeys{
  /handlers/.mem read/.style={
    \pgfkeyscurrentpath/.get key parts=\lit@key,
    /litmus/\lit@keyname/.style/.expanded={/litmus/mem event={\lit@keyname:#1}},
    /litmus/eiids order/.list append/.expand once=\lit@keyname,
  },
  /handlers/.mem write/.style={
    \pgfkeyscurrentpath/.get key parts=\lit@key,
    /litmus/\lit@keyname/.style/.expanded={/litmus/mem event={\lit@keyname:#1}},
    /litmus/eiids order/.list append/.expand once=\lit@keyname,
  },
  /handlers/.events/.style={
    \pgfkeyscurrentpath/.get key parts=\lit@key,
    /tikz/\lit@keyname/.pic/.expanded={
      \noexpand\path[pic actions]
        { [ /litmus/eiids order/.initial=,
            /litmus/#1,
            /litmus/events={\lit@keyname},
            /litmus/eiids order/.get=\noexpand\lit@temp@eiids,
            /litmus/event nodes/.initial=,
            /litmus/add event node/.style={/litmus/event nodes/.append={node[/litmus/########1]}},
            /litmus/add event node/.list/.expanded=\noexpand\lit@temp@eiids,
          ]
          \noexpand\pgfkeysvalueof{/litmus/event nodes}
        }
        node[/litmus/events box={\lit@keyname}] (\lit@keyname) {};
    },
    % The following .pic is used when we collect the eiids, it is not
    % used to draw anything.
    /tikz/\lit@keyname-eiids/.pic/.expanded={
      \noexpand\path
        { [ /litmus/eiids order/.initial=,
            /litmus/#1,
            /litmus/events={\lit@keyname},
            /litmus/eiids order/.get=\noexpand\lit@temp@eiids,
            /utils/exec={\noexpand\xdef\noexpand\lit@eiids{\noexpand\lit@temp@eiids}},
          ]
        } ;
    },
  },
}

\litmusset{
  events/.value required, % ioid
  events/.style={
    /tikz/local bounding box=lbbox,
    /tikz/node distance=1pt,
    /tikz/start chain=events going {/tikz/below=of \tikzchainprevious.south west, /tikz/anchor=north west},
    /litmus/event/.append style={/tikz/on chain=events},
    /litmus/.cd scope={
      every events/.try={#1},
      events #1/.try
    },
  },
  events box internal/.style={/tikz/inner sep=0},
  events box/.value required, % ioid
  events box/.style={
    /utils/exec={%
      \csxdef{lit@#1@tid}{\lit@tid}%
      \csxdef{lit@#1@row}{\the\pgfmatrixcurrentrow}%
    },
    /tikz/behind path,
    /tikz/fit=(lbbox),
    /litmus/events box internal,
    /litmus/.cd scope={
      every events box/.try={#1},
      events box #1/.try,
    },
  },
  show event label/.is flag=true,
  event label/.code={{\color{black}#1:}},
  event/.value required,% eiid
  event/.style={
    % 'on chain=events' is added by 'events' so that 'event' can also
    % be used not on the chain (e.g. hand written diagrams).
    /tikz/name={#1-text},
    /utils/exec={%
      \csxdef{lit@#1@tid}{\lit@tid}%
      \csxdef{lit@#1@row}{\the\pgfmatrixcurrentrow}%
    },
    /tikz/node font=\sffamily,
    % If the heights of the first event in each thread do not match the
    % thread headers will not be aligned.
    /tikz/text height={height("abcdefghijklmnopqrstuvwxyz ABCDEFGHIJKLMNOPQRSTUVWXYZ 0123456789 =:/[]")},
    /litmus/.cd scope={
      every event/.try={#1},
      event #1/.try,
    },
    /litmus/show event label/.if true=then {
      /tikz/append after command={
        node[%
          anchor=base east,
          align=right,
          /litmus/.cd scope={
            every event label/.try={#1},
            event label #1/.try,
          },
        ] (#1-label) at (#1-text.base west) {\litmusset{event #1/eiid/.get with default={\lit@eiid}{#1}, event label/.expand once=\lit@eiid}}
        node[%
          fit={(#1-label)(#1-text)},
          inner sep=0pt,
          /litmus/.cd scope={
            every event box/.try={#1},
            event box #1/.try,
          },
        ] (#1) {}
      },
    } else {
      /tikz/append after command={
        node[%
          fit={(#1-text)},
          inner sep=0pt,
          /litmus/.cd scope={
            every event box/.try={#1},
            event box #1/.try,
          },
        ] (#1) {}
      },
    },
  },
  event/.prefix style={/tikz/anchor=base west},
      % this has to be prefixed so that the 'anchor' of 'going' can
      % take precedence (for all nodes except the first one on the chain)
  init/.style={
    /tikz/name=init-#1,
    /tikz/circle,
    /tikz/minimum size=0.75ex,
    /tikz/inner sep=0,
    /tikz/draw=rf color,
    /tikz/fill=rf color,
    /tikz/node distance=0.75em,
    /tikz/above left=of #1,
    /litmus/.cd scope={
      every init/.try={#1},
      event init-#1/.try,
    },
  },
}

\litmusset{
  init right/.style={
    /litmus/event init-#1/.append style={/tikz/xshift=0.75em, /tikz/right=of #1},
    /litmus/(init-#1) rf (#1)/.append style=/tikz/swap,
  },
  init left/.style={
    /litmus/event init-#1/.append style={/tikz/xshift=-0.75em, /tikz/left=of #1},
  },
  init above right/.style={
    /litmus/event init-#1/.append style={/tikz/above right=of #1},
    /litmus/(init-#1) rf (#1)/.append style=/tikz/swap,
  },
  init below right/.style={
    /litmus/event init-#1/.append style={/tikz/below /tikz/right=of #1},
    /litmus/(init-#1) rf (#1)/.append style=/tikz/swap,
  },
  init above left/.style={
    /litmus/event init-#1/.append style={/tikz/above left=of #1},
  },
  init below left/.style={
    /litmus/event init-#1/.append style={/tikz/below left=of #1},
    /litmus/(init-#1) rf (#1)/.append style=/tikz/swap,
  },
}

\pgfkeys{
  /litmus/add check exists/.code args={event #1}{%
    \listgadd{\lit@check@exists@list}{#1}%
  },
  /handlers/.change eiid/.style args={to #1}{
    \pgfkeyscurrentpath/.get key parts=\change@temp,
    \change@temp/eiid/.initial={#1},
    /litmus/add check exists/.expand once=\change@tempname,
  },
  /handlers/.change value/.style args={from #1 to #2}{
    \pgfkeyscurrentpath/.get key parts=\change@temp,
    \change@temp/value/.initial={#2},
    \change@temp/value-expected/.initial={#1},
    /litmus/add check exists/.expand once=\change@tempname,
  },
  /handlers/.change addr/.style args={from #1 to #2}{
    \pgfkeyscurrentpath/.get key parts=\change@temp,
    \change@temp/fp addr/.initial={#2},
    \change@temp/fp addr-expected/.initial={#1},
    /litmus/add check exists/.expand once=\change@tempname,
  },
  /handlers/.change offset/.style args={from #1 to #2}{
    \pgfkeyscurrentpath/.get key parts=\change@temp,
    \change@temp/fp offset/.initial={#2},
    \change@temp/fp offset-expected/.initial={#1},
    /litmus/add check exists/.expand once=\change@tempname,
  },
  /handlers/.change size/.style args={from #1 to #2}{
    \pgfkeyscurrentpath/.get key parts=\change@temp,
    \change@temp/fp size/.initial={#2},
    \change@temp/fp size-expected/.initial={#1},
    /litmus/add check exists/.expand once=\change@tempname,
  },
  /handlers/.change footprint/.style args={from #1+#2/#3 to #4+#5/#6}{
    \pgfkeyscurrentpath/.get key parts=\change@temp,
    \change@temp/.change addr=from #1 to #4,
    \change@temp/.change offset=from #2 to #5,
    \change@temp/.change size=from #3 to #6,
  },
  /handlers/.change mem access/.style args={from #1+#2/#3=#4 to #5+#6/#7=#8}{
    \pgfkeyscurrentpath/.get key parts=\change@temp,
    \change@temp/.change addr=from #1 to #5,
    \change@temp/.change offset=from #2 to #6,
    \change@temp/.change size=from #3 to #7,
    \change@temp/.change value=from #4 to #8,
  },
}

\litmusset{
  record exists/.code={\listgadd{\lit@exists@list}{#1}},
  check value/.code 2 args={
    \edef\lit@temp@exval{\pgfkeysifdefined{/litmus/#1-expected}{\pgfkeysvalueof{/litmus/#1-expected}}{#2}}
    \edef\lit@temp@val{#2}
    \ifx\lit@temp@exval\lit@temp@val\else
      \PackageError{litmus-tikz-library}{'#1' does not match the expected value (\lit@temp@exval\space <> \lit@temp@val)}{}
    \fi
  },
  mem event/.style args={#1:#2 #3+#4/#5=#6}{
    % <eiid>:<mem_access> <mem-loc>+<offset>/<size>=<value>
    % e.g. 'a:W x+0/4=1'
    %      'a:Racq x+4/2=5'
    /litmus/record exists={#1},
    /litmus/check value={/litmus/event #1/fp addr}{#3},
    /litmus/check value={/litmus/event #1/fp offset}{#4},
    /litmus/check value={/litmus/event #1/fp size}{#5},
    /litmus/check value={/litmus/event #1/value}{#6},
    /litmus/event={#1},
    /litmus/event #1/fp addr/.get with default={\lit@addr}{#3},
    /litmus/event #1/fp offset/.get with default={\lit@offs}{#4},
    /litmus/event #1/fp size/.get with default={\lit@size}{#5},
    /litmus/event #1/value/.get with default={\lit@val}{#6},
    /tikz/node contents={%
      \litmusset{mem access/#2/.try, mem access/.retry={#2}}\,\litmusset{memfp/.expanded={\lit@addr}{\lit@offs}{\lit@size}}=\lit@val%
    },
  },
  mem access/.code={#1},
}

\litmusset{
  boxed events/.default=black!10,
  boxed events/.style={
    /litmus/events box internal/.append style={
      /tikz/rounded corners=2pt,
      /tikz/inner sep=2pt,
      /tikz/fill=#1,
    },
  },
}

\newif\iflit@set@vertical@align@to\lit@set@vertical@align@tofalse
\litmusset{
  horizontal align/.style={
    /tikz/to path={
      (\tikztostart) -- (\tikztotarget |- \tikztostart) \tikztonodes
    },
  },
  set vertical align to/.is global if flag=lit@set@vertical@align@to,
  vertical align to node/.global store in=\lit@vertical@align@to,
  vertical align to/.style={
    /litmus/set vertical align to/.if true=then {
      /litmus/vertical align to node={#1},
      /litmus/set vertical align to=false,
    },
  },
  % after this style is applied all the edges in scope will be vertically
  % aligned with #1 (default: the first node)
  vertical align/.default=\tikztostart,
  vertical align/.style={
    % HACK: clear 'edge distance {0,1}' to locally disable this routing
    /litmus/edge distance {0,1}/.style=,
    /litmus/set vertical align to=true,
    /tikz/to path={
      [/litmus/vertical align to/.expanded={#1}]
      (\tikztostart.south -| \lit@vertical@align@to) -- (\tikztotarget.north -| \lit@vertical@align@to) \tikztonodes
    },
  },
  instruction relations/.style={
    /litmus/vertical align,
    /litmus/.cd scope={
      every instruction relations/.try,
    },
    /tikz/execute at end scope/.value of=/litmus/add to instruction relations \lit@tid,
  },
  /handlers/.add instructions edge/.style={
    \pgfkeyscurrentpath/.get key parts=\lit@key,
    /litmus/add instructions edge/.expand once={\lit@keyname={#1}},
  },
  add instructions edge/.style args={(#1-#2) #3 (#4)=#5}{
    (#1-#2) #3 (#4)/.append style={#5},
    /litmus/add to instruction relations #1/.append={%
      \path (#1-#2) edge[/litmus/.cd scope={#3,#5}] (#4);%
    },
  },
  event relations/.style={
    /litmus/.cd scope={
      every event relations/.try,
    },
    /tikz/execute at end scope/.value of=/litmus/add to event relations,
  },
  /handlers/.add events edge/.style={
    \pgfkeyscurrentpath/.get key parts=\lit@key,
    /litmus/add events edge/.expand once={\lit@keyname={#1}},
  },
  add events edge/.style args={(#1) #2 (#3)=#4}{
    % in most cases it's ok to put #4 in the edge options below
    % (after the #2), but this will not work if there is 'disable edge routing'
    % in side #4, as this key will be evaluated after the routing has
    % already been done. Hence, the following line:
    (#1) #2 (#3)/.append style={#4},
    /litmus/add to event relations/.append={%
      \ifboolexpr{%
        test {\lit@if@init@event{#1}} and%
        not test {\lit@ifnodeexists{#1}}%
      }{% then
        \path node[gobble init/.style args={init-##1}{/litmus/init=##1},gobble init=#1] {} edge[/litmus/.cd scope={#2}] (#3);%
      }{% else
          \path (#1) edge[/litmus/.cd scope={#2}] (#3);%
      }%
    },
  },
}

\def\lit@if@init@event#1{\@lit@if@init@event#1-\par}
\def\@lit@if@init@event#1-#2\par{\ifstrequal{init}{#1}{\@firstoftwo}{\@secondoftwo}}

\litmusset{
  label pos/.initial=0.5,
  label text/.initial=\empty,
  relation/.style={
    /tikz/arrows=->,
    /tikz/thick,
    /tikz/edge node={
      node[
        auto,
        font=\sffamily,
        inner sep=1pt,
        /litmus/.cd scope={
          every label/.try,
          every #1 label/.try,
        },
        pos/.value of=/litmus/label pos,
        node contents/.value of=/litmus/label text,
      ]
    },
    /litmus/.cd scope={
      every relation/.try,
    },
  },
  hide/.style={
    /tikz/overlay, /tikz/to path=, /tikz/opacity=0, /tikz/text opacity=0,
    /tikz/fill opacity=0, /tikz/text height=0, /tikz/text depth=0,
    /tikz/text width=0, /tikz/inner sep=0, /tikz/outer sep=0,
    /tikz/minimum size=0,
  }, % 'to path=' affects 'vertical align' (the first node changes)
  hide edge/.style={/litmus/hide #1},
  hide edges/.style={/litmus/hide edge/.list={#1}},
}

\litmusset{%
  enable edge routing/.style={
    /litmus/edge routing/.style={
      /utils/ifboolexpe={%
        test {\ifcsdef{lit@\tikztotarget @tid}}%
        and test {\ifcsdef{lit@\tikztostart @tid}}%
        and test {\ifcsdef{lit@\tikztotarget @row}}%
        and test {\ifcsdef{lit@\tikztostart @row}}%
      } then {
        /tikz/evaluate={
          int \x,\y;
          \x = int(\csuse{lit@\tikztotarget @tid}) - int(\csuse{lit@\tikztostart @tid});
          \y = int(\csuse{lit@\tikztotarget @row}) - int(\csuse{lit@\tikztostart @row});
        },
        /litmus/edge distance {\x,\y}/.try,
      },
    },
  },
  disable edge routing/.style={
    /litmus/edge routing/.style=,
  },
  /litmus/edge routing/filter/.is family,
  /litmus/edge routing/.belongs to family=/litmus/edge routing/filter,
  /litmus/disable edge routing/.belongs to family=/litmus/edge routing/filter,
  /litmus/enable edge routing/.belongs to family=/litmus/edge routing/filter,
  /litmus/apply edge routing filter to/.code={%
    \pgfkeys{/pgf/key filters/active families/.install key filter}%
    \pgfqkeysactivatesinglefamilyandfilteroptions{/litmus/edge routing/filter}{/litmus}{#1}%
  },
  /handlers/.disable routing/.style={
    \pgfkeyscurrentpath/disable routing/.style={},
  },
  % Back edges:
  edge distance {-3,-1}/.style={/tikz/out=-160, /tikz/in=-55, /tikz/looseness=0.8, /litmus/label pos=0.9},
  edge distance {-2,-1}/.style={/tikz/out=-160, /tikz/in=-55, /litmus/label pos=0.8},
  edge distance {-1,-4}/.style={/tikz/out=-190, /tikz/in=-145, /tikz/looseness=1},
  edge distance {-1,-3}/.style={/tikz/out=-185, /tikz/in=-145, /tikz/looseness=1.2},
  edge distance {-1,-2}/.style={/tikz/out=-180, /tikz/in=-145, /tikz/looseness=1.5},
  edge distance {-1,-1}/.style={/tikz/out=-160, /tikz/in=-145, /tikz/looseness=2.1, /litmus/label pos=0.8},
  edge distance {0,-1}/.style={/tikz/bend left=60},
  % Loop or events from the same instruction:
  edge distance {0,0}/.style={/tikz/out=180, /tikz/in=180, /tikz/distance=1.5em, /tikz/swap},
  % Immediate successor: for such edges between ioids the 'vertical align'
  % overrides this routing; when such edge exists between eiids you probably
  % want to add 'hide po', '(<eiid1>) <edge> (<eiid2)/.disable routing'
  % and '(<eiid1>) <edge> (<eiid2)/.style={vertical align=<ioid>}' to the
  % litmus diagram.
  edge distance {0,1}/.style={/tikz/bend left=60},
  % Skip (forward) edges:
  edge distance {0,2}/.style={/tikz/bend left=60},
  edge distance {0,3}/.style={/tikz/bend left=60},
  edge distance {0,4}/.style={/tikz/bend left=55},
  edge distance {0,5}/.style={/tikz/bend left=55},
  % Strait horizontal line:
  edge distance {1,0}/.style={
    /tikz/to path={(\tikztostart.east) -- (\tikztotarget.west |- \tikztostart.east) \tikztonodes},
  },
}

% define basic colors
\definecolor{litmusBlack}     {rgb}{0,0,0}             % svgname Black
\definecolor{litmusBrown}     {rgb}{0.647,0.165,0.165} % svgname Brown
\definecolor{litmusRed}       {rgb}{1,0,0}             % svgname Red
\definecolor{litmusOrange}    {rgb}{1,0.648,0}         % svgname Orange
\definecolor{litmusIndigo}    {rgb}{0.294,0.000,0.510} % svgname Indigo
\definecolor{litmusDarkGreen} {rgb}{0.000,0.392,0.000} % svgname DarkGreen

\litmusset{
  /handlers/.litmus relation/.style={
    \pgfkeyscurrentpath/.get key parts=\@lit@key,
    /utils/exec={\expandafter\colorlet\expandafter{\@lit@keyname\space color}{#1}},
    \@lit@key/.default/.expand once=\@lit@keyname,
    \@lit@key/.estyle={
      /litmus/relation={\@lit@keyname},
      /litmus/label text={##1},
      /tikz/color={\@lit@keyname\space color},
      % We run the user defined styles with a special filter that only
      % allowes the '/litmus/{disable|enable} edge routing' keys to be
      % executed, as these keys have to be executed before we do the
      % edge routing and the other keys must be executed after
      /litmus/every \@lit@keyname/.belongs to family=/litmus/edge routing/filter,
      /litmus/(\noexpand\tikztostart) \@lit@keyname\space (\noexpand\tikztotarget)/.belongs to family=/litmus/edge routing/filter,
      /litmus/apply edge routing filter to={
        every \@lit@keyname/.try,
        (\noexpand\tikztostart) \@lit@keyname\space (\noexpand\tikztotarget)/.try,
      },
      % Try all the 'disable routing' options and if none of them work do the routing.
      /litmus/(\noexpand\tikztostart) \@lit@keyname\space (\noexpand\tikztotarget)/disable routing/.try,
      /litmus/(\noexpand\tikztostart)/disable routing/.retry,
      /litmus/(\noexpand\tikztotarget)/disable routing/.retry,
      /litmus/from (\noexpand\tikztostart)/disable routing/.retry,
      /litmus/to (\noexpand\tikztotarget)/disable routing/.retry,
      /litmus/\@lit@keyname/disable routing/.retry,
      /litmus/edge routing/.retry,
      %
      /litmus/.cd scope={
        every \@lit@keyname/.try,
        (\noexpand\tikztostart) \@lit@keyname\space (\noexpand\tikztotarget)/.try,
      },
    },
    /litmus/hide \@lit@keyname/.estyle={\@lit@key/.style={/litmus/hide}},
    % Add an inverse edge (not a real edge, just a wrapper that adds
    % \textsuperscript{-1} to the default label)
    \@lit@key^{-1}/.default/.expanded={\expandonce\@lit@keyname\noexpand\textsuperscript{-1}},
    \@lit@key^{-1}/.estyle={\@lit@key={##1}},
  },
  base language/.style={
    % Generated in pp.ml (pp_brief_write_kind):
    /litmus/mem access/W/.code={W},
    % Generated in pp.ml (pp_brief_read_kind):
    /litmus/mem access/R/.code={R},
    /litmus/po/.litmus relation=litmusBlack,
    /litmus/co/.litmus relation=litmusBrown,
    /litmus/rf/.litmus relation=litmusRed,
      /litmus/hide rf/.append style={/litmus/init/.style=/litmus/hide},
      /litmus/rf'/.style 2 args={/litmus/rf={\litmusset{memrf={##1}{##2}}}},
    /litmus/fr/.litmus relation=litmusOrange,
      /litmus/fr'/.style 2 args={/litmus/fr={\litmusset{memfr={##1}{##2}}}},
    /litmus/ctrl/.litmus relation=litmusIndigo,
    /litmus/ctrl+addr/.litmus relation=litmusIndigo,
    /litmus/addr/.litmus relation=litmusIndigo,
    /litmus/data/.litmus relation=litmusIndigo,
    % Edge with multiple labels
    /litmus/multi/.litmus relation=litmusBlack,
    /litmus/multi/separator/.store in=\lit@multi@sep,
    /litmus/multi/separator={,},
    /litmus/append multi label/.code={%
      \pgfkeysgetvalue{/litmus/##1/.@def}{\lit@temp@label@}%
      % Handle inverse edges like 'po^{-1}' (we only look for the ^)
      \def\lit@temp@p####1^####2\lit@term{####1}%
      \edef\lit@temp@label{\expandonce\lit@temp@label{\expandonce\lit@multi@sep}{\noexpand\color{\lit@temp@p##1^\lit@term\space color}\expandonce\lit@temp@label@}}%
    },
    /litmus/multi'/.style={
      /utils/exec={\def\lit@temp@label{\@gobble}},
      /litmus/append multi label/.list={##1},
      /litmus/multi/.expand once=\lit@temp@label,
    },
    %
    /litmus/.cd scope={
      every litmus base language/.try,
    },
  },
}

\litmusset{
  AArch64/.style={
    /litmus/listing options/.if defined=then {
      /litmus/listing options/.prefix={language=AArch64,}
    },
    /litmus/base language,
    % Generated in pp.ml (pp_brief_write_kind):
    /litmus/mem access/Wrel/.code={W\textsubscript{rel}},
    /litmus/mem access/Wexc/.code={W\textsubscript{exc}},
    /litmus/mem access/We-r/.code={W\textsubscript{e-r}},
    % Generated in pp.ml (pp_brief_read_kind):
    /litmus/mem access/Racq/.code={R\textsubscript{acq}},
    /litmus/mem access/Rwacq/.code={R\textsubscript{wacq}},
    /litmus/mem access/Rexc/.code={R\textsubscript{exc}},
    /litmus/mem access/Re-a/.code={R\textsubscript{e-a}},
    /litmus/mem access/Rnt/.code={R\textsubscript{nt}},
    /litmus/isb/.litmus relation=litmusIndigo,
    /litmus/ctrl+isb/.litmus relation=litmusIndigo,
    /litmus/dmb/.litmus relation=litmusDarkGreen,
    /litmus/dmb st/.litmus relation=litmusDarkGreen,
    /litmus/dmb ld/.litmus relation=litmusDarkGreen,
    /litmus/dsb/.litmus relation=litmusDarkGreen,
    /litmus/dsb st/.litmus relation=litmusDarkGreen,
    /litmus/dsb ld/.litmus relation=litmusDarkGreen,
    /litmus/.cd scope={
      every AArch64/.try,
    },
  },
}

\litmusset{
  Power/.style={
    /litmus/listing options/.if defined=then {
      /litmus/listing options/.prefix={language=Power,}
    },
    /litmus/base language,
    % Generated in pp.ml (pp_brief_write_kind):
    /litmus/mem access/WX/.code={W\textsubscript{X}},
    % Generated in pp.ml (pp_brief_read_kind):
    /litmus/mem access/RX/.code={R\textsubscript{X}},
    /litmus/isync/.litmus relation=litmusIndigo,
    /litmus/ctrl+isync/.litmus relation=litmusIndigo,
    /litmus/sync/.litmus relation=litmusDarkGreen,
    /litmus/lwsync/.litmus relation=litmusDarkGreen,
    /litmus/eieio/.litmus relation=litmusDarkGreen,
    /litmus/.cd scope={
      every Power/.try,
    },
  },
}

\litmusset{
  RISCV/.style={
    /litmus/listing options/.if defined=then {
      /litmus/listing options/.prefix={language=RISCV,}
    },
    /litmus/base language,
    % Generated in pp.ml (pp_brief_write_kind):
    /litmus/mem access/W.RCpc/.code={W\textsubscript{RCpc}},
    /litmus/mem access/W.RCsc/.code={W\textsubscript{RCsc}},
    /litmus/mem access/W.aqrl/.code={W\textsubscript{aqrl}},
    /litmus/mem access/Wcon/.code={W\textsubscript{con}},
    /litmus/mem access/Wcon.RCpc/.code={W\textsubscript{con.RCpc}},
    /litmus/mem access/Wcon.RCsc/.code={W\textsubscript{con.RCsc}},
    /litmus/mem access/Wcon.aqrl/.code={W\textsubscript{con.aqrl}},
    % Generated in pp.ml (pp_brief_read_kind):
    /litmus/mem access/R.RCpc/.code={R\textsubscript{RCpc}},
    /litmus/mem access/R.RCsc/.code={R\textsubscript{RCsc}},
    /litmus/mem access/R.aqrl/.code={R\textsubscript{aqrl}},
    /litmus/mem access/Rres/.code={R\textsubscript{res}},
    /litmus/mem access/Rres.RCpc/.code={R\textsubscript{res.RCpc}},
    /litmus/mem access/Rres.RCsc/.code={R\textsubscript{res.RCsc}},
    /litmus/mem access/Rres.aqrl/.code={R\textsubscript{res.aqrl}},
    /litmus/fence.i/.litmus relation=litmusIndigo,
    /litmus/ctrl+fence.i/.litmus relation=litmusIndigo,
    /litmus/fence rw rw/.litmus relation=litmusDarkGreen,
      /litmus/fence rw rw/.default={fence rw,rw},
    /litmus/fence rw r/.litmus relation=litmusDarkGreen,
      /litmus/fence rw r/.default={fence rw,r},
    /litmus/fence rw w/.litmus relation=litmusDarkGreen,
      /litmus/fence rw w/.default={fence rw,w},
    /litmus/fence r rw/.litmus relation=litmusDarkGreen,
      /litmus/fence r rw/.default={fence r,rw},
    /litmus/fence r r/.litmus relation=litmusDarkGreen,
      /litmus/fence r r/.default={fence r,r},
    /litmus/fence r w/.litmus relation=litmusDarkGreen,
      /litmus/fence r w/.default={fence r,w},
    /litmus/fence w rw/.litmus relation=litmusDarkGreen,
      /litmus/fence w rw/.default={fence w,rw},
    /litmus/fence w r/.litmus relation=litmusDarkGreen,
      /litmus/fence w r/.default={fence w,r},
    /litmus/fence w w/.litmus relation=litmusDarkGreen,
      /litmus/fence w w/.default={fence w,w},
    /litmus/fence.tso/.litmus relation=litmusDarkGreen,
    /litmus/.cd scope={
      every RISCV/.try,
    },
  },
}

\litmusset{
  x86/.style={
    /litmus/listing options/.if defined=then {
      /litmus/listing options/.prefix={language=x86,}
    },
    /litmus/base language,
    % Generated in pp.ml (pp_brief_write_kind):
    /litmus/mem access/Wlck/.code={W\textsubscript{lck}},
    % Generated in pp.ml (pp_brief_read_kind):
    /litmus/mem access/Rlck/.code={R\textsubscript{lck}},
    /litmus/MFENCE/.litmus relation=litmusDarkGreen,
    /litmus/.cd scope={
      every x86/.try,
    },
  },
}


\litmusset{
  state prefix font/.initial=,
  initial state prefix/.initial={Initial~state:\ },
  final state prefix/.initial={Final~state:\ },
  final state result/.initial=unknown,
  final state result/unknown/.style={/litmus/.cd scope={every final state unknown/.try}},
  final state result/allowed/.style={/litmus/.cd scope={every final state allowed/.try}},
  final state result/forbidden/.style={/litmus/.cd scope={every final state forbidden/.try}},
  final state allowed/.style={
    /litmus/final state result=allowed,
    /litmus/final state prefix={Allowed:\ },
    /utils/if no value={#1} then {} else {
      /litmus/final state={#1},
    },
    /litmus/expected observations=allowed,
  },
  final state forbidden/.style={%
    /litmus/final state result=forbidden,
    /litmus/final state prefix={Forbidden:\ },
    /utils/if no value={#1} then {} else {
      /litmus/final state={#1},
    },
    /litmus/expected observations=forbidden,
  },
  %
  table header/.code={
    \node[
      name=table header,
      node distance=0,
      inner sep=0,
      outer sep=0,
      align=left,
      text width=\lit@threads@width - (2*\pgfkeysvalueof{/pgf/inner xsep}) + \pgflinewidth,
      /litmus/litmus name/.get=\lit@name,
      /litmus/architecture/.get=\lit@arch,
      node contents={\rlap\lit@name\hfill\llap\lit@arch},
      /litmus/.cd scope={
        every table header/.try,
      },
      #1, % Should be positioning
    ];
  },
  initial state box/.code={
    \node[
      name=initial state box,
      outer sep=0,
      node distance=0,
      node contents={{\pgfkeysvalueof{/litmus/state prefix font}\pgfkeysvalueof{/litmus/initial state prefix}}\pgfkeysvalueof{/litmus/initial state}},
      text width=\lit@threads@width - (2*\pgfkeysvalueof{/pgf/inner xsep}),
      draw,
      /litmus/.cd scope={
        every box/.try,
        every state/.try,
        every initial state/.try,
      },
      #1, % Should be positioning
    ];
  },
  final state box/.code={
    \node[
      name=final state box,
      outer sep=0,
      node distance=0,
      node contents={{\pgfkeysvalueof{/litmus/state prefix font}\pgfkeysvalueof{/litmus/final state prefix}}\pgfkeysvalueof{/litmus/final state}},
      text width=\lit@threads@width - (2*\pgfkeysvalueof{/pgf/inner xsep}),
      draw,
      /litmus/.cd scope={
        every box/.try,
        every state/.try,
        every final state/.try,
      },
      /litmus/final state result/.selected,
      #1, % Should be positioning
    ];
  },
%   \TODO{change obs to Y,N and - in san serif, try \footnote, add colors to match the tables}
  expected observations/.initial=unknown,
  expected observations/allowed/.style={/litmus/.cd scope={expected observations allowed/.try}},
  expected observations/forbidden/.style={/litmus/.cd scope={expected observations forbidden/.try}},
  experimental result/.is choice,
  experimental result/observed/.style={
    /tikz/node contents=Y,
    /litmus/.cd scope={
      experimental result/every observed/.try,
    },
  },
  experimental result/not observed/.style={
    /tikz/node contents=N,
    /litmus/.cd scope={
      experimental result/every not observed/.try,
    },
  },
  experimental result/missing/.style={
    /tikz/node contents={--},
    /litmus/.cd scope={
      experimental result/every missing/.try,
    },
  },
  hw observations box/.code={
    \pgfkeysalso{/litmus/hw observations/.get=\lit@obs}%
    \ifdefempty{\lit@obs}{}{% else
      \begin{scope}[%
        /litmus/hw observations/.get=\lit@obs,
        start chain=observations going right,
        node distance=1pt,
        /litmus/expected observations/.selected,
      ]
        \node[
          name=hw observations prefix,
          outer sep=0,
          node distance=0,
          node contents={HW observations:},
          on chain,
          /litmus/.cd scope={
            every hw observations prefix/.try,
          },
          #1, % Should be positioning
        ];
        \foreach \o/\r in \lit@obs
          \node[
  %           anchor=base west,
  %           at={(hw observations prefix.base east)},
            on chain,
            minimum size=0, inner sep=0, outer sep=0,
            node contents=,
            /litmus/.cd scope={
              every hw observation/.try,
            },
            /utils/ifboolexpr={test {\ifdefstring{\r}{0}}} then {
              /litmus/experimental result=missing,
            } else {
              /utils/ifboolexpr={test {\ifdefstring{\o}{0}}} then {
                /litmus/experimental result=not observed,
              } else {
                /litmus/experimental result=observed,
              },
            },
  %           /litmus/experimental result/.expand once=\s,
          ];
      \end{scope}
    }%
  },
  table construction/.style={
    /litmus/table header={/tikz/above=1pt of threads box, /tikz/anchor=base},
    /litmus/initial state box={/tikz/below=2pt of threads box},
    /litmus/final state box={/tikz/below=of initial state box},
    /litmus/hw observations box={/tikz/anchor=north west,/tikz/at=(final state box.south west)},
  },
  read states file/.code={%
    \begingroup%
      \InputIfFileExists{#1.states.tex}{}{%
        \def\litmusname{??}%
        \def\litmusarch{??}%
        \def\rmemmodel{??}%
        \def\initstate{??}%
        \def\finalstate{??}%
      }%
      \InputIfFileExists{#1.hw.tex}{}{%
        \def\litmusobservations{}%
      }%
      \edef\lit@temp{%
    \endgroup%
        \noexpand\pgfkeysalso{%
          /litmus/litmus name/.initial={\litmusname},%
          /litmus/architecture/.initial={\litmusarch},%
          /litmus/rmem model/.initial={\rmemmodel},%
          /litmus/initial state/.initial={\initstate},%
          /litmus/final state/.initial={\finalstate},%
          /litmus/hw observations/.initial={\litmusobservations},%
        }%
      }\lit@temp%
    \pgfkeysalso{%
      % In most case, the relaxed model indicates the test is forbidden
      % and any other model indicates the test is allowed:
      /litmus/rmem model/.try selected,%
      /litmus/final state allowed/.retry,%
    }%
  },
}
\litmusset{
  litmus table/.style n args={4}{
    /litmus/litmus name/.initial={#1},
    /litmus/architecture/.initial={#2},
%     /litmus/rmem model/.initial={#??},
    /litmus/initial state/.initial={#3},
    /litmus/final state/.initial={#4},
    /litmus/@litmus table,
    /litmus/hw observations/.initial=,
  },
  @litmus table/.style={
    /tikz/execute at end scope={
      \begin{scope}[/litmus/.cd scope={every table scope/.try}]
        \begin{scope}[on background layer]
          % Fit a node for each assembly line (behind the borders)
          \tikzmath{
            coordinate \@topleft,\@botright;
            \@topleft = (threads-1.north west);
            \@botright = (threads-\chaincount{threads}.north east) - (0pt,\assemrowsep);
            int \@l;
            for \@l in {1,...,\maxassemlines}{
              {%
                \node[%
                  inner sep=0, minimum size=0,
                  yshift=-(\@l - 1) * (\assemrowsep),
                  fit=(\@topleft) (\@botright),
                  node contents=,
                  /litmus/.cd scope={
                    every code line/.try/.expand once=\@l,
                    evaluate={\lit@line@parity = isodd(\@l) ? "odd" : "even";},
                    every \lit@line@parity\space code line/.try/.expand once=\@l,
                    every code line \@l/.try,
                  },
                ];
              };
            };
          }
          % This node fits around all the assembly and thread headers
          \node[
            inner sep=0, outer sep=0,
            /utils/exec={\def\lit@temp@nodes{}},
            evaluate={
              int \c,\t,\tid;
              \c = int(\chaincount{threads});
              for \t in {1,...,\c}{
                \tid = \t - 1;
                print{%
                  \lit@ifnodeexists{thread \tid\space header}{% then
                    \xdef\lit@temp@nodes{\lit@temp@nodes (threads-\t) (thread \tid\space header)}%
                  }{% else
                    \xdef\lit@temp@nodes{\lit@temp@nodes (threads-\t)};
                  }%
                };%
              };
            },
            fit/.expand once=\lit@temp@nodes,
            draw,
            /litmus/.cd scope={
              every box/.try,
              every threads box/.try,
            },
          ] (threads box) {};
          % Draw lines between threads
          \tikzmath{
            int \c,\t;
            \c = int(\chaincount{threads});
            if \c > 1 then {
              for \t in {2,...,\c}{
                % do not remove the "extra" brackets (see \tikzmath 'print')
                {\path[draw, /litmus/.cd scope={every border between threads/.try}] (threads box.north -| threads-\t.west) -- (threads box.south -| threads-\t.west);};
              };
            };
          }
          % Draw a line between thread headers and assembly
          \path[draw, /litmus/.cd scope={every thread headers border/.try}]
            (threads box.west |- threads-1.north) -- (threads box.east |- threads-1.north);
        \end{scope}
        \tikzmath{
          coordinate \diff;
          \diff = (threads box.north east) - (threads box.south west);
          let \lit@threads@width = \diffx;
          let \lit@threads@height = \diffy;
        }
        % Draw the header, initial state and final state
        \litmusset{table construction}
      \end{scope}
    },
  },
}

\litmusset{
  rmem model/relaxed/.style={
    /litmus/final state forbidden,
  },
}

% \litmusset{
%   table shape/.is choice,
%   table shape/going right/.style={},
%   table shape/3/.style={},
%   table shape/4/.style={},
%   table shape=going right,
%
%   first thread below second/.style 2 args={
%     /tikz/evaluate={
%       let \tid = #1;
%       \ts = int(#1 + 1);
%       \below@ts = int(#2 + 1);
%       let \prev@ts = #1;
%     },
%     /litmus/instructions/.append style={
%       /tikz/evaluate if={\lit@tid == \tid} then {
%         below={4ex of threads-\below@ts.south west},
%       },
%       /tikz/evaluate if={\lit@tid == \tid + 1} then {
%         right=of threads-\prev@ts.north east,
%       },
%     },
%     /litmus/@litmus table/.append style={
%       /tikz/execute at end scope={
%         \begin{scope}[/litmus/.cd scope={every table scope/.try}]
%           \begin{scope}[on background layer]
%             \draw (threads-\ts.north west) -- (threads-\ts.north east);
%             \lit@ifnodeexists{thread \tid\space header}{% then
%               \draw (thread \tid\space header.north -| threads-\ts.west) -- (thread \tid\space header.north -| threads-\ts.east);
%             }{}%
%           \end{scope}
%         \end{scope}
%       },
%     },
%   },
% }

%%%%%%%%%%%%%%%%%%%%%%%%%%%%%%%%%%%%%%%%%%%%%%%%%%%%%%%%%%%%%%%%%%%%%%
% \input wrappers
% for all the following macros:
%   #1 - tikz options (optional)
%   #2 - mode ('events', 'assem' or 'full')
%   #3 - .tikz file without the suffix

% We write to this file litmus tests for which diagrams are needed
% Line format: "<context>:<arch>/<test name>"
\def\lit@opt@context{main}
\newwrite\litmusfigs@file
\immediate\openout\litmusfigs@file=\jobname.litmusfigs
% \closeout\litmusfigs@file

% We have to use this 'if' because putting the code inside \IfFileExists
% requires arguments in the tikz options to have extra #'s
\newif\iflit@file@exists

\litmusset{
  missing file/.style={
    draw=red, very thick,
    fill=red!25,
    font=\tiny\bf, align=left,
  },
}

\newcommand{\inputlitmus}[3][]{%
  \immediate\write\litmusfigs@file{\lit@opt@context:#3}%
  \IfFileExists{#3.tikz}{\lit@file@existstrue}{\lit@file@existsfalse}
  \iflit@file@exists%
    \begin{scope}[/litmus/.cd scope={kind={#2},#1}]%
      \input{#3.tikz}
    \end{scope}%
  \else%
    \PackageWarning{litmus-tikz-library}{missing file: #3.tikz}%
    \begin{scope}
      \node[/litmus/missing file] {??missing #3.tikz file??};%
    \end{scope}%
  \fi%
}

\newcommand{\inputlitmuspicture}[3][]{%
  \immediate\write\litmusfigs@file{\lit@opt@context:#3}%
  \IfFileExists{#3.tikz}{\lit@file@existstrue}{\lit@file@existsfalse}
  \iflit@file@exists%
    \begin{tikzpicture}[/litmus/.cd scope={every litmus picture/.try,kind={#2},#1}]%
      \input{#3.tikz}
    \end{tikzpicture}%
  \else%
    \PackageWarning{litmus-tikz-library}{missing file: #3.tikz}%
    \begin{tikzpicture}%
      \node[/litmus/missing file] {??missing #3.tikz file??};%
    \end{tikzpicture}%
  \fi%
}

\newcommand{\inputlitmustable}[3][]{%
  \inputlitmuspicture[%
    every litmus table/.try,%
    read states file={#3},%
    #1,%
    @litmus table,%
  ]{#2}{#3}%
}
